%!TEX program = pdflatex
\RequirePackage[l2tabu, orthodox]{nag}
\documentclass{article}

% FONTS
\usepackage[T1]{fontenc}

% Import MathDesign (this brings along Bitstream Charter)
% http://www.ctan.org/tex-archive/fonts/mathdesign/
\usepackage[bitstream-charter]{mathdesign}
\usepackage{amsmath}
\usepackage[scaled=0.92]{PTSans}
\usepackage{inconsolata}

% GEOMETRY
\usepackage[
  paper  = letterpaper,
  left   = 1.5in,
  right  = 1.5in,
  top    = 1.0in,
  bottom = 1.0in,
  ]{geometry}

% COLOR
\usepackage[usenames,dvipsnames]{xcolor}
\definecolor{shadecolor}{gray}{0.9}

% SPACING and TEXT
\usepackage[final,expansion=alltext]{microtype}
\usepackage[english]{babel}
\usepackage[parfill]{parskip}

% COUNTERS
\renewcommand{\labelenumi}{\color{black!67}{\arabic{enumi}.}}
\renewcommand{\labelenumii}{{\color{black!67}(\alph{enumii})}}
\renewcommand{\labelitemi}{{\color{black!67}\textbullet}}

% FIGURES
\usepackage{graphicx}
\usepackage[labelfont=bf]{caption}
\usepackage[format=hang]{subcaption}

% BIBLIOGRAPHY
\usepackage{natbib}

% HYPERREF
\usepackage[colorlinks,linktoc=all]{hyperref}
\usepackage[all]{hypcap}
\hypersetup{citecolor=Violet}
\hypersetup{linkcolor=black}
\hypersetup{urlcolor=MidnightBlue}

\newcommand{\red}[1]{\textcolor{BrickRed}{\textbf{#1}}}
\pagenumbering{gobble}

\begin{document}
\hfill{\today}

Journal of Machine Learning Research

Dear Editors:

I am writing to submit my manuscript entitled \emph{``Adaptive Data Augmentation for Thompson Sampling''} for consideration for publication in the \emph{Journal of Machine Learning Research}.

This manuscript \red{develops a novel adaptive data augmentation framework for Thompson Sampling}. The proposed framework addresses the limitations of the conventional ridge estimator in linear contextual bandits and \red{constructs a new estimator that accurately learns the rewards of all arms—including those not selected—without imposing restrictive assumptions on the context distribution}.

The central innovations of this work are \red{(i) the construction of a hypothetical linear bandit model specifically designed to facilitate efficient parameter learning with minimal augmentation samples}, and \red{(ii) the coupling of these augmented samples with those from the original problem instance}. The resulting algorithm \red{achieves a cumulative regret bound that matches the minimax lower bound up to logarithmic factors}.

In addition to these theoretical guarantees, the proposed method \red{demonstrates significantly improved empirical performance} compared to existing approaches in the literature. I believe that these contributions provide a new perspective on adaptive data augmentation and have the potential to advance the study of Thompson Sampling for linear contextual bandits.

I suggest the following action editors and referees for my submission.

Action Editors:
\begin{itemize}
    \item \red{Alekh Agarwal, Google (alekhagarwal@google.com)}
    %\item \red{Arindam Banerjee, UIUC (arindamb@illinois.edu)}
    \item \red{Kevin Jamieson, University of Washington (jamieson@cs.washington.edu)}
    \item \red{Wouter Koolen, CWI (wmkoolen@cwi.nl)}
    %\item \red{Tor Lattimore, DeepMind (tor.lattimore@gmail.com)}
    \item \red{Alexandre Proutiere, KTH Royal Institute of Technology (alepro@kth.se)}
    \item \red{Zhengyuan Zhou, DeepMind (tor.lattimore@gmail.com)}
\end{itemize}
Reviewers:
\begin{itemize}
    \item \red{Dylan Foster, Microsoft Research (dylanfoster@microsoft.com)}
    \item \red{Botao Hao, OpenAI (haobotao000@gmail.com)}
    \item \red{Branislav Kveton, Adobe Research (kveton@adobe.com)}
    \item \red{Akshay Krishnamurthy, Microsoft Research (akshaykr@microsoft.com)}
    %\item \red{Lev Reyzin, UIC (lreyzin@uic.edu)}
    \item \red{Johannes Kirschner, Swiss Data Science Center (johannes.kirschner@sdsc.ethz.ch)}
\end{itemize}

This submission has the following keywords: \red{Thompson Sampling}, \red{minimax optimal regret bound}, \red{hypothetical bandit problem},  \red{adaptive data augmentation}, \red{coupling}

I confirm that I do not have a conflict of interest with the action editors and referees I suggest above. 
Further, I consent to my submission of this manuscript to the \emph{Journal of Machine Learning Research}.

I appreciate your time and consideration and I look forward to your feedback.

Sincerely,

\medskip

\red{Wonyoung Kim (Chung-Ang University)}\\
\end{document}






































